\documentclass[a4paper,10pt,titlepage]{article}
\usepackage{graphicx} % Required for inserting images
\usepackage[margin=2cm]{geometry} % Adjust the margins
\usepackage{amsmath}
\usepackage{amssymb}
\numberwithin{equation}{subsection}
\usepackage{float}
\usepackage{bm}

\title{
	\textbf{\huge{Modern Design of Control Systems}}\\
	\textit{Notes and Summary}}

\author{Gregorio Valenti}

\date{\textit{Polytechnic of Turin}, September 2024 - February 2025}

\begin{document}
	\maketitle

	\section{Feedback Control System Definition}
	\hspace{1cm}
	For this course, we will consider the model below to represent the system under study:
	
	\begin{figure}[H] % Use [H] to fix the position
		\centering
		\includegraphics[scale=1]{images/feedback_loop_description.png}
		\caption{Feedback Control System}
		\label{fig:image1}
	\end{figure}

	With the following characteristics:
	\begin{enumerate}
		\item[$\bullet$] \textbf{Plant} $G_p$ with disturbance $d_p$
		\item[$\bullet$] \textbf{Actuator} $G_a$ with disturbance $d_a$
		\item[$\bullet$] \textbf{Sensor} $G_s$ with disturbance $d_s$
		\item[$\bullet$] \textbf{Feedback Controller} $G_c$ 	
	\end{enumerate}
	
	\section{Step response of prototype $\boldmath{2}^{\textbf{nd}}$ order system}
	For a prototype second order system with \textbf{Sensitivity} and \textbf{Complementary sensitivity} functions of the form:
	\begin{equation}
		T(s) = \dfrac{1}{1+\dfrac{2\zeta}{\omega_n}s+\dfrac{s^2}{\omega_n^2}} \quad ; \quad S(s) = \dfrac{s\left(\dfrac{2\zeta}{\omega_n}+\dfrac{s}{\omega_n^2}\right)}{1+\dfrac{2\zeta}{\omega_n}s+\dfrac{s^2}{\omega_n^2}}
	\end{equation}
	
	
	From the transient response indices for maximum overshoot $\hat{s}$, rise time $t_r$ and settling time $t_{s,\alpha\%}$, we can compute the system's damping coefficient, as well as constraints on the natural frequency and resonance peaks $T_p$ and $S_p$:
	\begin{enumerate}
		\item[$\bullet$] $\displaystyle \zeta = \frac{\left| \log (\hat{s}) \right|}{\sqrt{\pi^2 + \log^2 (\hat{s})}} \hfill (2.0.2)$
		\item[$\bullet$] $\displaystyle \omega_{n,tr} = \frac{1}{t_r\sqrt{1-\zeta^2}} \cdot (\pi - acos(\zeta)) \quad; \quad
		\omega_{n,t,s\alpha \%} = \frac{\log(\frac{100}{\alpha})}{t_{s,\alpha \%}\zeta} \hfill (2.0.3)$
		\item[$\bullet$] $\displaystyle T_p = \frac{1}{2\zeta \sqrt{1-\zeta^2}} \quad; \quad S_p = \frac{2\zeta \sqrt{2+4\zeta^2+2\sqrt{1+8\zeta^2}}}{\sqrt{1+8\zeta^2}+4\zeta^2-1} \hfill (2.0.4)$
	\end{enumerate}
	
	\section{Steady-state Response to Polynomial Reference Inputs}
	Requirements on steady-state tracking error and steady-state errors in the presence of polynomial disturbances can be translated into constraints on the Sensitivity function.
	Therefore, by first writing $S(s) = s^{\nu+p}S^*(s)$, we can apply the final value theorem to each expression as described in the following subsections.
	\subsection{Tracking Error}
	\subsubsection{Problem formulation}
	The \textbf{tracking error} is defined as:
	\begin{equation}
		e_r(t) = y_d(t)-y_r(t) = K_dr(t)-y_r(t)
	\end{equation}
	Applying the \textbf{final value theorem} leads to:
	\begin{align}
		\left| e_r^\infty \right| &= \displaystyle\lim_{t\to\infty} 	\left|e_r(t)\right| = \displaystyle\lim_{s\to0}s \left|e_r(s)\right| = \displaystyle\lim_{s\to0}s \left|K_dr(s)-y_r(s)\right| = \nonumber \\
		&= \displaystyle\lim_{s\to0}s \left|G_{re}(s)r(s)\right| = 	\displaystyle\lim_{s\to0}s \left|S(s)K_dr(s)\right| = \nonumber \\
		&= \displaystyle\lim_{s\to0}s 	\left|s^{\nu+p}S^*(s)K_d\dfrac{h!R_0}{s^{h+1}}\right| =
		\begin{cases}
			0, & \text{if } \nu+p>h \\
			\left|S^*(0)K_dh!R_0\right|, & \text{if } \nu+p = h
		\end{cases}
	\end{align}
	The given system type is then $\nu+p$.
	\subsubsection{Conclusions}
	Considering the specification $\left|e_r^\infty\right| \leq \rho_r  $, we obtain the following result:
	\begin{enumerate}
		\item[$\bullet$] case $\rho_r=0 \implies S(s)$ must have a zero at $s=0$ with multiplicity greater than $h \implies$ We have no constraint on $\left|S^*(0)\right| $ 
		\item[$\bullet$] case $\rho_r>0 \implies S(s)$ must have a zero at $s=0$ with multiplicity $h \implies$ We have the following constraint:
		\begin{equation}
			\left|S^*(0)\right| \leq \dfrac{\rho_r}{K_dh!R_0}	
		\end{equation}
	\end{enumerate}
	
	\subsection{Error due to Generic Polynomial Disturbances $\bm{d(t)}$}
	\subsubsection{Problem formulation}
	The \textbf{output error} due to the generic disturbance $d(t)$ is the contribution of the disturbance to the output $y(t)$:
	\begin{equation}
		e_d(t) = y_d(t)
	\end{equation}
	
	\subsubsection{Polynomial disturbance $\bm{d_a(t)}$}
	Applying the \textbf{final value theorem} leads to:
	\begin{align}
		\left| e_{d_a}^\infty \right| &= \displaystyle\lim_{t\to\infty} 	\left|e_{d_a}(t)\right| = \displaystyle\lim_{s\to0}s \left|e_{d_a}(s)\right| = \displaystyle\lim_{s\to0}s \left|y_{d_a}(s)\right| = \displaystyle\lim_{s\to0}s \left| S(s)G_p(s)d_a(s) \right| = \nonumber \\
		&= \displaystyle\lim_{s\to0} 	\left|s^{\nu+p+1}S^*(s)G_p(s)\dfrac{h!D_{a0}}{s^{h+1}}\right| = \displaystyle\lim_{s\to0} \left|s^{\nu+1}S^*(s)K_p\dfrac{h!D_{a0}}{s^{h+1}}\right| =
		\begin{cases}
			0, & \text{if } \nu>h \\
			\left|S^*(0)K_ph!D_{a0}\right|, & \text{if } \nu = h
		\end{cases}
	\end{align}
	We then obtain two cases as before:
	\begin{enumerate}
		\item[$\bullet$] case $\rho_a=0 \implies S(s)$ must have a zero at $s=0$ with multiplicity greater than $h \implies$ We have no constraint on $\left|S^*(0)\right| $ 
		\item[$\bullet$] case $\rho_a>0 \implies S(s)$ must have a zero at $s=0$ with multiplicity $h \implies$ We have the following constraint:
		\begin{equation}
			\left|S^*(0)\right| \leq \dfrac{\rho_a}{K_ph!D_{a0}}	
		\end{equation}
	\end{enumerate}
	
	\subsubsection{Polynomial disturbance $\bm{d_p(t)}$}
	Applying the \textbf{final value theorem} leads to:
	\begin{align}
		\left| e_{d_p}^\infty \right| &= \displaystyle\lim_{t\to\infty} 	\left|e_{d_p}(t)\right| = \displaystyle\lim_{s\to0}s \left|e_{d_p}(s)\right| = \displaystyle\lim_{s\to0}s \left|y_{d_p}(s)\right| = \displaystyle\lim_{s\to0}s \left| S(s)d_p(s) \right| = \nonumber \\
		&= \displaystyle\lim_{s\to0}	\left|s^{\nu+p+1}S^*(s)\dfrac{h!D_{p0}}{s^{h+1}}\right| =
		\begin{cases}
			0, & \text{if } \nu+p>h \\
			\left|S^*(0)h!D_{p0}\right|, & \text{if } \nu+p = h
		\end{cases}
	\end{align}
	We then obtain two cases as before:
	\begin{enumerate}
		\item[$\bullet$] case $\rho_p=0 \implies S(s)$ must have a zero at $s=0$ with multiplicity greater than $h \implies$ We have no constraint on $\left|S^*(0)\right| $ 
		\item[$\bullet$] case $\rho_p>0 \implies S(s)$ must have a zero at $s=0$ with multiplicity $h \implies$ We have the following constraint:
		\begin{equation}
			\left|S^*(0)\right| \leq \dfrac{\rho_p}{h!D_{p0}}	
		\end{equation}
	\end{enumerate}
	
	\section{Steady-state Response to Sinusoidal Disturbances}
	\subsection{Output disturbance $\bm{d_p(t)}$}
	The focus is on the class of sinusoidal signals of the following type:
	\begin{equation}
		d_p = a_p\sin(\omega_pt) \quad \forall\omega_p\leq\omega_p^+ \quad \text{given } a_p \text{ and } w_p^+
	\end{equation}
	The output at steady-state is required to be bounded by a given constant:
	\begin{equation}
		\left| e_{d_p}^\infty \right| = \left| y_{d_p}^\infty \right| \leq \rho_p \quad \text{with } \rho_p>0
	\end{equation}
	The specification leads to a \textbf{frequency domain constraint} on the Sensitivity function $S(s)$ which can be computed as follows:
	\begin{align}
		\left| e_{d_p}^\infty \right| &= \left| y_{d_p}^\infty \right| = \left| a_pS(j\omega_p)\sin(\omega_pt+\varphi_p) \right| \leq a_p\left|S(j\omega_p)\right| \leq \rho_p \nonumber \\
		&\implies \left|S(j\omega_p)\right| \leq \dfrac{\rho_p}{a_p} = M_S^{LF} \quad \forall\omega_p \leq \omega_p^+
	\end{align}
	
	\subsection{Sensor Noise $\bm{d_s(t)}$}
	The focus is on the class of sinusoidal signals of the following type:
	\begin{equation}
		d_s = a_s\sin(\omega_st) \quad \forall\omega_s\geq\omega_s^- \quad \text{given } a_s \text{ and } w_s^-
	\end{equation}
	The output at steady-state is required to be bounded by a given constant:
	\begin{equation}
		\left| e_{d_s}^\infty \right| = \left| y_{d_s}^\infty \right| \leq \rho_s \quad \text{with } \rho_s>0
	\end{equation}
	The specification leads to a \textbf{frequency domain constraint} on the Complementary Sensitivity function $T(s)$ which can be computed as follows:
	\begin{align}
		\left| e_{d_s}^\infty \right| &= \left| y_{d_s}^\infty \right| = \left| a_sT(j\omega_s)\dfrac{1}{G_s}\sin(\omega_st+\varphi_s) \right| \leq a_s\left|T(j\omega_s)\dfrac{1}{G_s}\right| \leq \rho_s \nonumber \\
		&\implies \left|T(j\omega_s)\right| \leq \dfrac{\rho_sG_s}{a_s} = M_T^{HF} \quad \forall\omega_s \leq \omega_s^-
	\end{align}
	
	
	\section{Weighting Functions Construction}
	\subsection{Rational Approximation of Frequency Constraints}
	\begin{enumerate}
		\item[$\bullet$] Rational functions of the Laplace variable s are used to approximate the frequency domain constraints on $S(s)$ and $T(s)$.
		\item[$\bullet$] The parameters of the approximating functions (steady-state gain zeros and poles) can be moved to get the desired result.
		\item[$\bullet$] \textbf{Butterworth polynomials} can be used either as denominator or numerator of the approximating rational function to effectively retain constraints on different frequency ranges.
	\end{enumerate}
	\begin{table}[H] % Use [H] to fix position
		\begin{minipage}{0.52\textwidth} % Adjust width as needed
			\centering
			\begin{tabular}{|c|c|}
				\hline
				\textbf{Polynomial Order} & \textbf{Polynomial Structure} \\ \hline
				0 & $1$ \\ \hline
				1 & $1+\dfrac{s}{\omega_a}$ \\ \hline
				2 & $1+\dfrac{2\zeta}{\omega_a}s+\left(\dfrac{s}{\omega_a}\right)^2$ \\ \hline
				3 & $1+\dfrac{2}{\omega_a}s+2\left(\dfrac{s}{\omega_a}\right)^2+\left(\dfrac{s}{\omega_a}\right)^3$ \\ \hline
			\end{tabular}
			\caption{Butterworth Polynomials}
			\label{tab:polynomial_table}
		\end{minipage}
		\hfill
		\begin{minipage}{0.5\textwidth}
			\raggedright
			\vspace{-54pt}
			The \textbf{key property} of Butterworth polynomials is that, when used in the numerator or denominator of a rational function, they increase or decrease the frequency response magnitude by 3 dB respectively, at the frequency $\omega_a$, regardless of the polynomial's order.
		\end{minipage}
	\end{table}
	The goal is to obtain the \textbf{rational functions} $\bm{W_S^{-1}(s)}$ \textbf{and} $\bm{W_T^{-1}(s)}$ such that the constraints derived above are satisfied.
	\begin{enumerate}
		\item[$\bullet$] Considering low frequencies we have:
		\begin{equation}
			\left| W_S^{-1}(j\omega) \right| \leq M_S^{LF} \quad \forall 	\omega_p\leq\omega_p^+ \quad ; \quad \max_\omega \left| W_S^{-1}(\infty) \right| \leq S_{p0}
		\end{equation}	
		\item[$\bullet$] Considering high frequencies we have:
		\begin{equation}
			\left| W_T^{-1}(j\omega) \right| \leq M_T^{HF} \quad \forall 	\omega_s\geq\omega_s^- \quad ; \quad \max_\omega \left| W_T^{-1}(j\omega) \right| \leq T_{p0} \implies \left| W_T^{-1}(0) \right| = T_{p0}
		\end{equation}
	\end{enumerate}
	
	\begin{figure}[htbp]
		\centering
		% First minipage for the first image
		\begin{minipage}{0.45\textwidth} % Adjust the width as needed
			\centering
			\includegraphics[width=\linewidth]{images/Approximation_W_S.png}
			\caption{Weighting Function $W_S^{-1}$}
			\label{fig:image2}
		\end{minipage}
		\hfill % Adds horizontal space between the images
		% Vertical line
		\vrule width 0.5pt % Adjust the width of the line as needed
		\hfill % Adds horizontal space before the second image
		% Second minipage for the second image
		\begin{minipage}{0.45\textwidth} % Adjust the width as needed
			\centering
			\vspace{0pt}
			\includegraphics[width=1.135\linewidth]{images/Approximation_W_T.png}
			\caption{Weighting Function $W_T^{-1}$}
			\label{fig:image3}
		\end{minipage}
	\end{figure}
	
	\section{Performance Specification as $\bm{H_\infty}$ Norm Constraints}
	\subsection{$\bm{H_\infty}$ Norm Definition}
	By defining:
	\begin{enumerate}
		\item[$\bullet$] $W_S(s)$ as the inverse of the rational approximation of the frequency domain constraints on the Sensitivity function S(s)
		\item[$\bullet$] $W_T(s)$ as the inverse of the rational approximation of the frequency domain constraints on the Complementary Sensitivity function T(s)
	\end{enumerate}
	Design constraints obtained from the considered performance requirements can be written in the following compact form:
	\begin{equation}
		\left| W_S(j\omega)S(j\omega) \right| \leq 1 \quad ; \quad \left| W_T(j\omega)T(j\omega) \right| \leq 1 \quad \forall\omega	
	\end{equation}
	The $\bm{H_\infty}$ \textbf{norm} of a SISO LTI system with transfer function $H(s)$ is defined as:
	\begin{equation}
		\left\lVert H(s) \right\rVert_\infty \triangleq \max_\omega \left| H(j\omega) \right|
	\end{equation}
	It is now possible to rewrite the design constraints obtained above in terms of the weighted $H_\infty$ norm of $S(s)$ and $T(s)$:
	\begin{equation}
		\left\lVert W_S(s)S(s) \right\rVert_\infty \leq 1 \quad ; \quad \left\lVert W_T(s)T(s) \right\rVert_\infty \leq 1
	\end{equation}

	\section{Unstructured Uncertainty Modelling}
	Mathematical models cannot exactly describe a physical process, irrespective of their complexity. Thus, model uncertainty has to be taken into account when a mathematical model is used to analyse the behaviour of a system, or to design a feedback control system.
	\textbf{Unstructured uncertainty} comes into play when complete ignorance regarding the order and the phase behaviour of the system is assumed, and parametric uncertainty can also be described by means of unstructured model sets.
	\subsection{Uncertainty Model Sets}
	The following four uncertainty model sets will be considered, in which:
	\begin{enumerate}
		\item[$\bullet$] $G_p(s) $ is the transfer function of the generic member of the uncertainty set. 
		\item[$\bullet$] $G_{pn}(s)$ is the transfer function of the \textbf{nominal model}.
		\item[$\bullet$] $\Delta(s)$ represents any possible transfer function whose $H_\infty$ norm is less than 1.
		\item[$\bullet$] $W_u(s)$ is a \textbf{weighting function} which accounts for the size of the uncertainty.
	\end{enumerate} 
	Particular focus will be on the second set.
	\subsubsection{Additive Uncertainty}
	The additive uncertainty model set is defined as:
	
	\begin{equation}
		M_a = \left\{ G_p(s): G_p(s)=G_{pn}(s)+W_u(s)\Delta(s), \left\lVert\Delta(s)\right\rVert_\infty \leq1 \right\}
	\end{equation}
	where, by construction, the following condition must be satisfied by the weighting function $W_u(s)$:
	\begin{equation}
		\left\lVert \dfrac{G_p(s)-G_{pn}(s)}{W_u(s)} \right\rVert_\infty = \left\lVert \Delta(s) \right\rVert_\infty \leq 1
	\end{equation}
	which is equivalent to:
	\begin{equation}
		\left| G_p(j\omega)-G_{pn}(j\omega) \right| \leq \left| W_u(j\omega) \right| \quad \forall \omega
	\end{equation}
	
	Below is an example of the frequency response of $W_u(j\omega)$ for an additive uncertainty model set:
	
	\begin{figure}[H] % Use [H] to fix the position
		\centering
		\includegraphics[width=0.6\textwidth]{images/additive_uncertainty.png}
		\caption{Weighting Function $W_u$ considering $M_a$}
		\label{fig:image6}
	\end{figure}
	
	\raggedright
	\subsubsection{Multiplicative Uncertainty}
	The multiplicative uncertainty model set is defined as:
	\begin{equation}
		M_m = \left\{ G_p(s): G_p(s)=G_{pn}(s) \left[1+W_u(s)\Delta(s)\right], \left\lVert\Delta(s)\right\rVert_\infty \leq1 \right\}
	\end{equation}
	where, by construction, the following condition must be satisfied by the weighting function $W_u(s)$:
	\begin{equation}
		\left\lVert \left( \dfrac{G_p(s)}{G_{pn}(s)}-1 \right) \dfrac{1}{W_u(s)}\right\rVert_\infty = \left\lVert \Delta(s) \right\rVert_\infty \leq 1
	\end{equation}
	which is equivalent to:
	\begin{equation}
		\left| \dfrac{G_p(j\omega)}{G_{pn}(j\omega)}-1 \right| \leq \left| W_u(j\omega) \right| \quad \forall \omega
	\end{equation}
	
	Below is an example of the frequency response of $W_u(j\omega)$ for a multiplicative uncertainty model set:
	
	\begin{figure}[H] % Use [H] to fix the position
		\centering
		\includegraphics[width=0.68\textwidth]{images/multiplicative_uncertainty.png}
		\caption{Weighting Function $W_u$ considering $M_m$}
		\label{fig:image7}
	\end{figure}
	
	\raggedright
	Now we consider the problem of properly selecting the \textbf{nominal model} $\bm{G_{pn}}$ in order to \textbf{minimize} the size $W_u$ of the unstructured uncertainty of the given model:
	Let's consider, without loss of generality, the following nominal model:
	\begin{equation}
		G_{pn}(s) = \dfrac{K_n}{s-p}
	\end{equation}
	where $K_n$ is a constant value to be computed.
	The weighting function $W_u(s)$ must satisfy the following condition:
	\begin{equation}
		\left\lVert \Delta(s) \right\rVert_\infty = \sup_\omega \left| \left( \dfrac{G_p(j\omega)}{G_{pn}(j\omega)}-1 \right) \dfrac{1}{W_u(j\omega)} \right| \leq 1
	\end{equation}
	which is equivalent to:
	\begin{equation}
		\left| W_u(j\omega) \right| \geq \left| \dfrac{G_p(j\omega)}{G_{pn}(j\omega)}-1 \right| = \left| \dfrac{K}{K_n}-1 \right| \quad \forall\omega,\;\forall K 
	\end{equation}
	$K_n$ can be selected in order to minimize the size of the uncertainty in the following way:
	\begin{equation}
		\left| W_u(j\omega) \right| \geq \min_{K_n}\max_K \left| \dfrac{K}{K_n}-1 \right| \quad \forall\omega
	\end{equation}
	It can be easily shown that:
	\begin{equation}
		\min_{K_n}\max_K \left| \dfrac{K}{K_n}-1 \right| = \min_{K_n}\max \left\{\ \left| \dfrac{\overline{K}-K_n}{K_n} \right|, \left| \dfrac{\underline{K}-K_n}{K_n} \right| \right\}
	\end{equation}
	The solution is found at the intersection of the two functions to maximise (between curly braces in 7.1.11), and it is the following:
	\begin{equation}
		K_n = \dfrac{\overline{K}+\underline{K}}{2}
	\end{equation}
	
	\subsubsection{Inverse Additive Uncertainty}
	The inverse additive uncertainty model set is defined as:
	\begin{equation}
		M_{ia} = \left\{ G_p(s): G_p(s)= \dfrac{G_{pn}(s)}{1+W_u(s)\Delta(s)G_{pn}(s)}, \left\lVert\Delta(s)\right\rVert_\infty \leq1 \right\}
	\end{equation}
	\subsubsection{Inverse Multiplicative Uncertainty}
	The inverse multiplicative uncertainty model set is defined as:
	\begin{equation}
		M_{im} = \left\{ G_p(s): G_p(s)= \dfrac{G_{pn}(s)}{1+W_u(s)\Delta(s)}, \left\lVert\Delta(s)\right\rVert_\infty \leq1 \right\}
	\end{equation}
	
	\subsection{Conclusion and Remarks}
	An important remark is that \textbf{unstructured uncertainty model sets can only provide a conservative description of parametric uncertainties} since, as shown in the above example, a complex function $\Delta(s)$ is used to account for the source of uncertainty, which is a real number.
	The unstructured uncertainty model set describes, at each frequency $\omega$, the uncertainty as a disk of radius $\left| W_u(j\omega)L_n(j\omega) \right|$.
	
	\section{Unstructured Uncertainty Modelling and Robustness}
	The aim is to study the stability of a feedback control system under the assumption that $G_p$ is an uncertain system described by a given uncertainty model set. The following discussion will consider a multiplicative uncertainty model set $M_m$.
	
	\subsection{Robust Stability}
	The feedback control system displayed in Figure 1 is \textbf{robustly stable} if and only if it is internally stable for each $G_p$ which belongs to $M_m$.
	As a result, the following condition on the weighting function $W_u$ must be satisfied:
	
	\begin{equation}
		\left\lVert W_uT_n \right\rVert_\infty < 1
	\end{equation} 
	
	where $T_n$ is the \textbf{nominal complementary sensitivity function}.
	Robust stability conditions for every described uncertainty model set are shown in the table below:
	\begin{center}
		\vspace{2pt}
		\begin{tabular}{|c|c|}
			\hline
			\textbf{Uncertainty Model Set} & \textbf{Robust Stability Condition} \\ \hline
			$M_m$ & $\left\lVert W_uT_n \right\rVert_\infty < 1$ \\ \hline
			$M_a$ & $\left\lVert W_uG_cS_n \right\rVert_\infty < 1$ \\ \hline
			$M_{ia}$ & $\left\lVert W_uG_{pn}S_n \right\rVert_\infty < 1$ \\ \hline
			$M_{im}$ & $\left\lVert W_uS_n \right\rVert_\infty < 1$ \\ \hline
		\end{tabular}
	\end{center}
	
	\subsection{Nominal Performance}
	The nominal performance conditions, derived previously in subsection 6.1, are requirements affecting the sensitivity and complementary sensitivity functions as follows:
	
	\begin{equation}
		\left\lVert W_SS_n \right\rVert_\infty < 1 \iff \left| 1+L_n(j\omega) \right| > \left| W_S(j\omega) \right| \quad \forall\omega 
	\end{equation}
	\begin{equation}
		\left\lVert W_TT_n \right\rVert_\infty < 1 
	\end{equation}
	
	\subsection{Robust Performance}
	The feedback system guarantees \textbf{robust performance} if and only if performance requirements are satisfied for each $G_p$ which belongs to the given uncertainty set.
	Considering the case in which performance requirements affect only the sensitivity function and uncertainty is described by means of a multiplicative uncertainty model set, the following condition must be satisfied:
	
	\begin{equation}
		\left\lVert \left| W_SS_n \right| + \left| W_uT_n \right | \right\rVert_\infty < 1
	\end{equation}
	
	\section{$\bm{H_\infty}$ Design for Robust Control}
	
	
	
	
\end{document}